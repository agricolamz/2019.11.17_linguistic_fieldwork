\documentclass[13pt, t]{beamer}
% Presento style file
\usepackage{config/presento}

% custom command and packages
% custom packages
\usepackage{textpos}
\setlength{\TPHorizModule}{1cm}
\setlength{\TPVertModule}{1cm}

\newcommand\crule[1][black]{\textcolor{#1}{\rule{2cm}{2cm}}}



\usepackage{color, colortbl}
\setlength{\columnseprule}{0.4pt} 

\title{\Large \hspace{-0.5cm} Полевая работа в лингвистике}
\author[shortname]{Георгий Мороз}
\institute[shortinst]{Международная лаборатория языковой конвергенции, НИУ ВШЭ}
\date{\begin{center} {\large 18 ноября 2019}  \end{center}}

\begin{document}

\begin{frame}[plain]
\maketitle
\end{frame}

\framecard[colorblue]{{\color{colorwhite} \Large Какие группы ездят?}}
\framepic{images/01_big_company.JPG}{\vspace{-0.2cm} Экспедиция НИУ ВШЭ в Мегеб (Дагестан) в 2014 году}
\framepic{images/02_small_company.JPG}{\vspace{-0.2cm} Маша Сапожникова в Кот-д’Ивуар в 2015 году}

\framecard[colorblue]{{\color{colorwhite} \Large Где живут?}}
\framepic{images/03_house.JPG}{\vspace{-0.2cm} Экспедиция в селение Кина (Дагестан) в 2019 году}
\framepic{images/04_chum.JPG}{\vspace{-0.2cm} Экспедиция в село Амгуэма (Чукотского АО) в 2016 году}
\framepic{images/05_family.JPG}{\vspace{-0.2cm} Лиза Востокова  (Гватемала) в 2016 году}
\framepic{images/06_school.jpg}{\vspace{-0.2cm} \color{colorwhite} Экспедиция в аул Блечепсин (Адыгея) в 2015 году}
\framepic{images/07_flat.jpg}{\vspace{-0.2cm} Экспедиция в Новосибирск в 2019 году}

\framecard[colorblue]{{\color{colorwhite} \Large Что делают?}}
\begin{frame}{Что делают?}
\begin{itemize}
\item \pause документируют языки (собирают словари и тексты) \pause 
\item исследуют грамматику \pause
\begin{itemize}
\item \textit{три хурмы}, \textit{четыре хурмы}, \textit{пять хур\dots}
\item \textit{вчера ты пылесосил, а сегодня я пылесо\dots} \pause
\end{itemize}
\item исследуют фонетику \pause
\begin{itemize}
\item где ставиться ударение в слове X?
\item какие звуки вы произносите в слове \textit{тостер}?
\end{itemize}
\item исследуют социолингвистику \pause
\begin{itemize}
\item где на каком языке говорят? 
\item какой язык используют в той или иной ситуации? 
\item какой уровень многоязычия у носителей в том или ином месте?
\end{itemize}
\pause
\item собирают материал для построение автоматического распознавания/синтеза речи
\item \dots
\end{itemize}
\end{frame}

\framecard[colorblue]{{\color{colorwhite} \Large Как работают?}}
\framepic{images/08_notebook.jpg}{\vspace{-0.2cm} Экспедиция в селение Гельмец (Дагестан) в 2016 году}
\framepic{images/09_laptop.png}{\vspace{-0.2cm} Экспедиция в аул Ходзь (Адыгея) в 2016 году}
\framepic{images/10_audio.jpg}{\vspace{-0.2cm} Экспедиция в аул Ходзь (Адыгея) в 2016 году}
\framepic{images/11_video.JPG}{\vspace{-0.2cm} Экспедиция в Новосибирск в 2019 году}
\framepic{images/12_phone.JPG}{\vspace{-0.2cm} Аня Клезович в экспедиции в Новосибирск в 2019 году}

\framecard[colorblue]{{\color{colorwhite} \Large Какие бывают проблемы?}}
\begin{frame}{Какие бывают проблемы?}
\begin{itemize}
\item в некоторые места трудно и дорого добираться \pause
\item в некоторых местах люди неохотно работают
\begin{itemize}
\item религия, культура, традиции
\item неудачное время экспедиции, все заняты сбором урожая, туристами и т. п.
\item стесняются: ``Я ничего не знаю! Поговорите со стариками!'' \pause
\end{itemize}
\item неправильные ожидания самих носителей: в основном, мы не учим язык, мы его изучаем \pause
\item условия для некоторых типов работы неподходящие:
\begin{itemize}
\item нельзя работать с молодыми без присутствия взрослых, а те начинают их исправлять
\item носитель плохо справляется с заданием, и все время спрашивает других
\item в помещении громко, эхо и т. п. и не получается сделать хорошую видео/аудиозапись
\item у носителей бывают представления о ``правильном'' языке, которые им мешают выполнять задания \pause
\end{itemize}
\item в некоторых местах мешают сложившиеся отношения между носителями или носителями и экспедиционерами
\end{itemize}
\end{frame}

\framecard[colorblue]{{\color{colorwhite} \Large Если что, пишите!\bigskip\\
agricolamz@gmail.com}}

\end{document}
